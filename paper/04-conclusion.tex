\section{Discussion}
\label{sec:conclusion}

In this paper, we introduced the $Beta\_SSE$ score, a measure of market inefficiency based on the unbiasedness regressions framework. The score follows patterns 
of well established market inefficiencies, such as the dot-com bubble, the 2008 financial crisis, and the COVID-19 pandemic.
We find that the market is not uniformly efficient, but instead goes through cycles of inefficiency followed by reversion to efficiency.
Over the last decade, inefficiency has been increasing, suggesting that the market has become less adept at using available information for present value discounts for future forecasts.
This rising inefficiency is correlated with the widening value spread, indicating a strong link between the weakness of the value premium and market inefficiency.
Causality between the two is not established, and can be investigated in a future work. However, this paper compliments the intuition 
presented in \citet{asness_2024} that the value premium has been weakening over the last decade because of market inefficiency.
A time series of market inefficiency is useful, since if market inefficiency can be forecasted it has implications for risk management.
Portfolios looking to profit from the value premium may want to consider the prevailing level of market inefficiency when constructing their portfolios,
adjusting their exposure during periods of inefficiency.