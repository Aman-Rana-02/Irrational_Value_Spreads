\section{Discussion}
\label{sec:conclusion}
\subsection{Summary}
In this paper we construct a measure of market efficiency using \citet{fama_EMH}'s definition of informational efficiency. Our measure is based on unbiasedness
regressions, and captures where the market has inefficient expectations. We build on \citet{asness_2024}'s case on inefficient
value spreads by showing a high correlation of market inefficiency and value spreads in the last decade.

\subsection{Drifting Inefficiency}
We learn the market's ability to present value future prices has weakened over the last decade. The cause of which is worth investigating
in future work. Either the market has been fundamentally more inefficient due to some microstructure, or the world could be in a regime of such innovation and technological change that the market has been systematically forced to revise its
present values of future expectations with accelerating error. This may not be entirely far-fetched, the last decade has seen the rise of the internet,
the proliferation of smartphones, and the rise of machine learning. These are all innovations that have changed the way we live and work, learn and build.
It is obvious the rate of innovation is accelerating, but is the market correctly pricing the higher moment, the rate of innovation acceleration?
Perhaps the markets haven't reached the level of sophistication to correctly price the current accelerating rate of innovation, indicating that
the rate of innovation is a systematic surprise and the cause of $Beta_/SSE$ drift. 

\subsection{Limitations and Future Work}
A limitation of this paper is that we have not established causality between market inefficiency and value spreads. This is a topic for future work.
They are clearly correlated in the last decade, but we cannot say that one causes the other. It is possible that 
growth stocks' high expectations are reasonable, value spreads are reflecting this new regime of expectations, and our co-moving inefficiency
is because of the unprecedented (Therefore difficult to price) nature of these innovations. \citet{asness_2024} argues otherwise, showing possible explanations and dismissing them, but he caveats that this
is not a complete proof. An implication of their co-movement is on portfolio management, if value spreads do widen during periods of inefficiency, portfolio managers with risk-premia mandates may choose to underweigh their value exposure during inefficient regimes, and lever back up once 
markets tend back to efficiency. Conversely, anomalies such as momentum should outperform during periods of informational inefficiency as the 
market comes to grips with unexpected information flows.

In future work we hope to investigate the decomposition of market inefficiency into under-reaction and over-reaction components.
There is an existing literature on the market's tendency to overreact and the robustness of our methodology would be strengthened if it aligns with
the current state of the literature. Additionally, since our measure has a forward-looking bias, attempting to forecast $Beta\_SSE$ would make it 
more practical for its use in portfolio and risk management.