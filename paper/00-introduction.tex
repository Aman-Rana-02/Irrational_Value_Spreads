\section{Introduction}
\indent \citet{fama_random_walk}'s efficient market hypothesis propsed that markets incorporate all information rationally and instantaneously. While 
this is an extreme assumption, it is established that markets tend to some form of efficiency. In an efficient market, investors are compensated for taking on systematic risk, therefore, understanding the level of 
market efficiency can be used to asses whether market participants will be rewarded fairly for the risks they bear.

"Value" in classical factor investing refers to stocks with low price-to-book ratios as per \citet{fama_french_1993} being considered undervalued. Being exposed to the Value factor means
going long on stocks that exhibit an undervalued nature, and short on stocks that exhibit an overvalued nature in order to capture the value premium; compensation for the risk of holding undervalued companies. 
Its relationship with market efficiency is critical: if markets are efficient, an investor should be rewarded for risk. 
However, when inefficiencies arise, value stocks may become mispriced, with the market willing to pay more for already 'expensive' stocks, and
less for already 'cheap' stocks. In these periods, investors may consider that the definition of value has changed, the subjective measure of 
'expensive' and 'cheap' has shifted. We can define a value spread as the ratio of the value of expensive stocks to the value of cheap stocks.
In this paper we show that generally, periods of inflated value spreads are correlated with an increase in inefficiency,
therefore we suggest that abnormal value spreads may not be reflective of a fundamental change in the market, but rather a mispricing.

Existing methods for measuring market efficiency typically focus on testing whether stock prices follow a random walk
\citep{fama_random_walk} \citep{lim_brooks_2010}. However these methods to not represent the variation of efficiency over time.
A more dynamic approach is needed to reflect the market’s changing behavior.

\citet{boguth_2023} uses unbiasedness regressions to examine price informativeness around FOMC meetings. 
This method visually shows us how prices adjust to information flow, in speed, and under or overreaction, using $R^2$s and $\beta$s over the unbiasedness regression window,
which is useful in an approach for measuring efficiency.

Our estimand is Market Inefficiency, which we define as the relative level of the markets ability to price in future information.
Building on this, we introduce the \textit{$Beta\_SSE$} estimate, which measures market efficiency at a point in time. 
By applying unbiasedness regressions within a rolling window, and constructing the \textit{$Beta\_SSE$} representation, we create a time series that tracks the relative level of market 
efficiency. \newline
\newline
Our findings show that:
\begin{itemize}
    \item The market is not uniformly efficient. Instead, it experiences cycles of inefficiency followed by reversion to efficiency.
    \item Over the last decade, inefficiency has been increasing, suggesting that the market has become less adept at incorporating available information into prices on longer time-scales.
    \item This rising inefficiency is correlated with the widening value spread, indicating a strong link between the weakness of the value premium and market inefficiency.
\end{itemize}

The rest of the paper is structured as follows: Section \ref{sec:data} describes the data used in this analysis. Section \ref{sec:market_efficiency} outlines the methodology used to measure market inefficiency.
Section \ref{sec:results} presents the results of our analysis. Section \ref{sec:conclusion} concludes and discusses the implications of our findings. 
Section \ref{sec:appendix} provides additional details on the methodology.